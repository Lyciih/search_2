
\usepackage{xparse}






\newcommand{\stack}[7]{
	% 1:圖原點x
	% 2:圖原點y
	% 3:外框高度
	% 4:外框寬度
	% 5:內部高度
	% 6:內部內縮
	% 7:內部個數

		\draw (#1, #2) -- (#1, #2 + #3);
		\draw (#1, #2) -- (#1 + #4, #2);
		\draw (#1 + #4, #2) -- (#1 + #4, #2 + #3);

		%\node at (#2/2,-#8) [font=\fontsize{#7}{14}\selectfont]  {#6};

		\ifthenelse{\equal{#7}{0}}{
			%等於0時不畫內部
		}{
			\draw (#1 + #6, #6) -- (#1 + #4 - #6, #6);
			\foreach \i in{1,...,#7}{
				\draw (#1 + #6, \i*#5 + #6) -- (#1 + #4 - #6, \i*#5 + #6);
				\draw (#1 + #6, \i*#5 + #6) -- (#1 + #6, \i*#5 + #6 - #5);
				\draw (#1 + #4 - #6, \i*#5 + #6) -- (#1 + #4 - #6, \i*#5 + #6 - #5);
			}
		}
}



\newcommand{\stackName}[6]{
	% 1:圖原點x
	% 2:圖原點y
	% 3:外框寬度
	% 4:標題距離
	% 5:標題大小
	% 6:標題內容

		\node at (#1 + #3/2, #2 - #4) [font=\fontsize{#5}{14}\selectfont]  {#6};
}


\newcommand{\stackArrow}[7]{
	% 1:圖原點x
	% 2:圖原點y
	% 3:外框寬度
	% 4:內部高度
	% 5:內部內縮
	% 6:指向目標
	% 7:箭頭長度


		\draw[->] (#1 + #3 + #7, #2 + #5 + #6*#4 - #4 + #4/2) -- (#1 + #3/2, #2 + #5 + #6*#4 -#4 + #4/2);
}

\newcommand{\queue}[7]{
	% 1:圖原點x
	% 2:圖原點y
	% 3:外框長度
	% 4:外框高度
	% 5:內部寬度
	% 6:內部內縮
	% 7:內部個數

	
		\draw (#1, #2) -- (#1 + #3, #2);
		\draw (#1, #2 + #4) -- (#1 + #3, #2 + #4);
		%\node at (#1/2, -#8) [font=\fontsize{#7}{14}\selectfont]  {#6};
		
		\ifthenelse{\equal{#7}{0}}{
			%等於0時不畫內部
		}{
			\draw (#1 + #6, #2 + #6) -- (#1 + #6, #2 + #4 - #6);
			\foreach \i in{1,...,#7}{
				\draw (#1 + #6 + \i*#5, #2 + #6) -- (#1 + #6 + \i*#5, #2 + #4 - #6);
				\draw (#1 + #6 + \i*#5, #2 + #6) -- (#1 + #6 + \i*#5 - #5, #2 + #6);
				\draw (#1 + #6 + \i*#5, #2 + #4 - #6) -- (#1 + #6 + \i*#5 - #5, #2 + #4 - #6);
			}
		}
}


\newcommand{\queueName}[6]{
	% 1:圖原點x
	% 2:圖原點y
	% 3:外框長度
	% 4:標題距離
	% 5:標題大小
	% 6:標題內容

		\node at (#1 + #3/2, #2 - #4) [font=\fontsize{#5}{14}\selectfont]  {#6};
}


\newcommand{\queueArrow}[7]{
	% 1:圖原點x
	% 2:圖原點y
	% 3:外框高度
	% 4:內部寬度
	% 5:內部內縮
	% 6:指向目標
	% 7:箭頭長度

	
		\draw[->] (#1 + #5 + #6*#4 - #4 + #4/2, #2 + #3 - #5 + #7) -- (#1 + #5 + #6*#4 - #4 + #4/2, #2 + #3/2);
}
