\documentclass{report}


% 註釋模組
\usepackage{comment}

% 中文設定
\usepackage{xeCJK}
    \setCJKmainfont[AutoFakeBold=3]{標楷體}
    \XeTeXlinebreaklocale "zh"             
    \XeTeXlinebreakskip = 0pt plus 1pt
    \setCJKmonofont{標楷體} % 设置 CJK 字体族,避免使用 verbatim 時會報錯

% 更多顏色設定
\usepackage{xcolor}
    \definecolor{lightgray}{rgb}{0.98, 0.98, 0.98}

% 可以貼程式碼,並且有語法高亮
\usepackage{listings}
    % 設定程式碼區塊的樣式
    \lstset{
        language=C++,
        backgroundcolor=\color{lightgray},
        frame=single,
        numbers=left,
        captionpos=b,
        breaklines=true
    }
    \renewcommand{\lstlistingname}{程式碼}


    % 添加 JavaScript 語言定義
    \lstdefinelanguage{JavaScript}{
    morekeywords={break, case, catch, continue, debugger, default, delete, do, else, finally, for, function, if, in, instanceof, new, return, switch, this, throw, try, typeof, var, void, while, with},
    morecomment=[s]{/*}{*/},
    morecomment=[l]//,
    morestring=[b]',
    morestring=[b]"
}

% 浮動體包
\usepackage{float}

%引用 geometry 模組 設定文件邊界為 2cm 大小為 A4
\usepackage[margin = 2cm, a4paper]{geometry}

%引用 setspace 模組
\usepackage{setspace}
    %設定行距為2倍
    \doublespacing
    %設定段落間距為 6 pt
    \setlength{\parskip}{6pt}
    
%引用 moresize 模組,提供更多字體大小選項
\usepackage{moresize}

% 引入 caption 宏包
\usepackage{caption}
    % 將圖片標籤的 Figure 改為 圖
    \captionsetup[figure]{name={圖}}  

\usepackage{graphicx} % Required for inserting images

\usepackage{tikz}
    \usetikzlibrary{positioning}
    \usetikzlibrary{matrix}
    \usetikzlibrary{arrows}
\usepackage{ifthen}


\usepackage{xparse}






\newcommand{\stack}[7]{
	% 1:圖原點x
	% 2:圖原點y
	% 3:外框高度
	% 4:外框寬度
	% 5:內部高度
	% 6:內部內縮
	% 7:內部個數

		\draw (#1, #2) -- (#1, #2 + #3);
		\draw (#1, #2) -- (#1 + #4, #2);
		\draw (#1 + #4, #2) -- (#1 + #4, #2 + #3);

		%\node at (#2/2,-#8) [font=\fontsize{#7}{14}\selectfont]  {#6};

		\ifthenelse{\equal{#7}{0}}{
			%等於0時不畫內部
		}{
			\draw (#1 + #6, #6) -- (#1 + #4 - #6, #6);
			\foreach \i in{1,...,#7}{
				\draw (#1 + #6, \i*#5 + #6) -- (#1 + #4 - #6, \i*#5 + #6);
				\draw (#1 + #6, \i*#5 + #6) -- (#1 + #6, \i*#5 + #6 - #5);
				\draw (#1 + #4 - #6, \i*#5 + #6) -- (#1 + #4 - #6, \i*#5 + #6 - #5);
			}
		}
}



\newcommand{\stackName}[6]{
	% 1:圖原點x
	% 2:圖原點y
	% 3:外框寬度
	% 4:標題距離
	% 5:標題大小
	% 6:標題內容

		\node at (#1 + #3/2, #2 - #4) [font=\fontsize{#5}{14}\selectfont]  {#6};
}


\newcommand{\stackArrow}[7]{
	% 1:圖原點x
	% 2:圖原點y
	% 3:外框寬度
	% 4:內部高度
	% 5:內部內縮
	% 6:指向目標
	% 7:箭頭長度


		\draw[->] (#1 + #3 + #7, #2 + #5 + #6*#4 - #4 + #4/2) -- (#1 + #3/2, #2 + #5 + #6*#4 -#4 + #4/2);
}

\newcommand{\queue}[7]{
	% 1:圖原點x
	% 2:圖原點y
	% 3:外框長度
	% 4:外框高度
	% 5:內部寬度
	% 6:內部內縮
	% 7:內部個數

	
		\draw (#1, #2) -- (#1 + #3, #2);
		\draw (#1, #2 + #4) -- (#1 + #3, #2 + #4);
		%\node at (#1/2, -#8) [font=\fontsize{#7}{14}\selectfont]  {#6};
		
		\ifthenelse{\equal{#7}{0}}{
			%等於0時不畫內部
		}{
			\draw (#1 + #6, #2 + #6) -- (#1 + #6, #2 + #4 - #6);
			\foreach \i in{1,...,#7}{
				\draw (#1 + #6 + \i*#5, #2 + #6) -- (#1 + #6 + \i*#5, #2 + #4 - #6);
				\draw (#1 + #6 + \i*#5, #2 + #6) -- (#1 + #6 + \i*#5 - #5, #2 + #6);
				\draw (#1 + #6 + \i*#5, #2 + #4 - #6) -- (#1 + #6 + \i*#5 - #5, #2 + #4 - #6);
			}
		}
}


\newcommand{\queueName}[6]{
	% 1:圖原點x
	% 2:圖原點y
	% 3:外框長度
	% 4:標題距離
	% 5:標題大小
	% 6:標題內容

		\node at (#1 + #3/2, #2 - #4) [font=\fontsize{#5}{14}\selectfont]  {#6};
}


\newcommand{\queueArrow}[7]{
	% 1:圖原點x
	% 2:圖原點y
	% 3:外框高度
	% 4:內部寬度
	% 5:內部內縮
	% 6:指向目標
	% 7:箭頭長度

	
		\draw[->] (#1 + #5 + #6*#4 - #4 + #4/2, #2 + #3 - #5 + #7) -- (#1 + #5 + #6*#4 - #4 + #4/2, #2 + #3/2);
}




\usepackage{pgfplots}
    \pgfplotsset{compat=1.18}


\title{tikz}
\author{YI-CHENG LI}
\date{November 2023}






\begin{document}

\maketitle

%產生目錄
\renewcommand{\contentsname}{目錄}
\tableofcontents

%產生圖目錄
\renewcommand{\listfigurename}{圖目錄}
\listoffigures

%產生程式碼目錄
\renewcommand{\lstlistlistingname}{程式碼目錄}
\lstlistoflistings 
%換頁
\clearpage




\chapter{背景知識}
\section{消息隊列}
\section{javascript事件循環}


\begin{figure}[h]
	\begin{tikzpicture}


		\node[draw, rectangle] at (0,0) {123};


		\stack{5cm}{0cm}{2cm}{1cm}{0.3cm}{0.75mm}{5}
		\stackName{5cm}{0cm}{1cm}{0.3cm}{8}{調用棧}
		\stackArrow{5cm}{0cm}{1cm}{0.3cm}{0.75mm}{2}{0.5cm}
		
		\queue{7cm}{1cm}{3cm}{0.5cm}{0.3cm}{0.75mm}{3}
		\queueName{7cm}{1cm}{3cm}{0.3cm}{8}{隊列}
		\queueArrow{7cm}{1cm}{0.5cm}{0.3cm}{0.75mm}{1}{0.5cm}
	\end{tikzpicture}
\end{figure}

\section{Web Workers}
javascript 原本的設計是基於單線程實現的,HTML5提供了Web Workers API 使 javascript 也可以用運用多線程的能力
javascript 對於異步任務或是平行運算是基於事件循環來實現的,其機制如下
\clearpage

\chapter{運算伺服器}

運算伺服器是整個繪圖系統的核心,保存著系統當前所有的圖形資料及狀態,根據收到的指令,對圖形進行增刪查改等操作。運算伺服器引用了CGAL(Computational Geometry Algorithms Library )幾何算法庫,用來對圖形進行幾何運算。

幾何運算包含非常多種的演算法,每種演算法所消耗的時間都不一樣,而用戶何時使用何種算法也是不確定的,為了避免整個系統在運算伺服器產生堵塞,需要對其進行解藕設計,在此我們利用Linux系統內建的消息隊列來實現,如圖~\ref{運算伺服器的架構}。
\begin{figure}[!h]
    \centering
    \begin{tikzpicture}
    
        \node[name=接收進程, draw, align=center] at (0,0) {接收};
    
        \matrix (input) [
        matrix of nodes, 
        right=1cm of 接收進程,
        nodes in empty cells, 
        nodes={draw, minimum size=0.3cm}
        ] {
          &   &   &   &   &   &   &   &  \\
        };
        \node[above=0.2cm] at (input.north) {輸入隊列};
    
        \node[name=幾何計算, draw, align=center, right=1cm of input] {幾何計算};
    
        \matrix (output) [
        matrix of nodes, 
        right=1cm of 幾何計算,
        nodes in empty cells, 
        nodes={draw, minimum size=0.3cm}
        ] {
          &   &   &   &   &   &   &   &  \\
        };
        \node[above=0.2cm] at (output.north) {輸出隊列};
    
        \node[name=送出進程, draw, align=center, right=1cm of output] {發送};


        \draw[->, shorten <=2mm,  shorten >=2mm] (接收進程.east) -- (input-1-1.west);
        \draw[->, shorten <=2mm,  shorten >=2mm] (input-1-9.east) -- (幾何計算.west);
        \draw[->, shorten <=2mm,  shorten >=2mm] (幾何計算.east) -- (output-1-1.west);
        \draw[->, shorten <=2mm,  shorten >=2mm] (output-1-9.east) -- (送出進程.west);
    
    \end{tikzpicture}
    \caption{運算伺服器的架構}
    \label{運算伺服器的架構}
\end{figure}

接收、運算、發送,共會用到三個進程,為了便於管理及使用,使用一個主進程啟動,再分裂出三個子進程,如圖~\ref{運算伺服器的進程分裂機制}。分裂完成後,主進程進入休眠狀態,當其中一個子進程崩潰時,主進程會被喚醒,進行錯誤處理或是重新產生崩潰的子進程。

% 運算伺服器的進程分裂機制
\begin{figure}[!h]
    \centering
    \begin{tikzpicture}
    
        \node[name=主進程, draw, align=center] {主進程};
        \node[name=運算進程, draw, align=center, below=of 主進程] {運算進程};
        \node[name=接收進程, draw, align=center, left=of 運算進程] {接收進程};
        \node[name=發送進程, draw, align=center, right=of 運算進程] {發送進程};

        \draw[->, shorten <=0mm,  shorten >=2mm] (主進程.south) -- (接收進程.north);
        \draw[->, shorten <=0mm,  shorten >=1mm] (主進程.south) -- (運算進程.north);
        \draw[->, shorten <=0mm,  shorten >=2mm] (主進程.south) -- (發送進程.north);

    
    \end{tikzpicture}
    \caption{運算伺服器的進程分裂機制}
    \label{運算伺服器的進程分裂機制}
\end{figure}

% 運算伺服器的子進程分裂
\begin{figure}
  \centering
    \begin{lstlisting}[language=C++, caption={運算伺服器的子進程分裂}, showstringspaces=false,, basicstyle=\tiny,  label=code:運算伺服器的子進程分裂]
#include "flow_compute_server.hpp"

int main(){
    
    pid_t main_process_ID = getpid();
    std::cout << "我是 main_process 進程: " << main_process_ID << std::endl ;


    pid_t rece_process_ID = fork();	//產生接收進程

    if (rece_process_ID == -1) {
        std::cerr << "rece_process fork 失敗" << std::endl;
        return 1;
    }
    else if (rece_process_ID == 0) {
        std::cout << "我是 rece_process 進程: " << getpid() << std::endl ;
        return 0;
    }
    else {
        pid_t comp_process_ID = fork();	//產生運算進程
        if (comp_process_ID == -1) {
            std::cerr << "comp_process fork 失敗" << std::endl;
            return 1;
        }
        else if (comp_process_ID == 0) {
            std::cout << "我是 comp_process 進程: " << getpid() << std::endl ;
            return 0;
        }
        else {
            pid_t send_process_ID = fork();	//產生發送進程
            if (send_process_ID == -1) {
                std::cerr << "send_process fork 失敗" << std::endl;
                return 1;
            }
            else if (send_process_ID == 0) {
                std::cout << "我是 send_process 進程: " << getpid() << std::endl ;
                return 0;
            }
        }
    }
    return 0;
}
    \end{lstlisting}
\end{figure}


\begin{figure}
    \begin{lstlisting}
    ./flow_compute_server
    我是 main_process 進程: 18107
    我是 rece_process 進程: 18108
    我是 comp_process 進程: 18109
    我是 send_process 進程: 18110
    \end{lstlisting}
    \caption{程式碼 \protect\ref{code:運算伺服器的子進程分裂}的輸出結果}
\end{figure}









\chapter{使用者介面}
為了能提供跨平台,並且不須額外進行安裝的特性,本系統的使用者介面將在瀏覽器上實現,並且使用最新的webgpu技術來繪製圖形。然而,因為WebGPU尚未普及,許多平台及瀏覽器都還只支援WebGL。為了能夠兼容這樣的情況,我們需要同時實作WebGPU和WebGL兩種繪圖機制,以便在需要時進行切換。

\section{檢查 WebGPU 和 WebGL 是否支援}
在網頁加載完成之後,首先需要檢查兩種圖形API的支援情況,在此實作兩個檢查函數,並模組化,在需要時引入使用。

% 兩種圖形API的檢查函數
\begin{figure}[ht]
  \centering
    \begin{lstlisting}[language=JavaScript, caption={圖形API支援檢查函數}, showstringspaces=false, basicstyle=\tiny,  label=code:圖形API支援檢查函數]
    
//檢查 WebGPU 是否支援的函數
export function webgpu_support_check(){
        if(navigator.gpu){
                console.log("WebGPU is support");
                return 1;
        }
        else{
                console.log("WebGPU is not support");
                return 0;
        }
}


//檢查 WebGL 是否支援的函數
export function webgl_support_check(){
        if(!!window.WebGLRenderingContext){
                console.log("WebGL is support");
                return 1;
        }
        else{
                console.log("WebGL is not support");
                return 0;

        }

    \end{lstlisting}
\end{figure}







\end{document}
